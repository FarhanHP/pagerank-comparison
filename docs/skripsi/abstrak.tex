\chapter*{ABSTRAK}

\textbf{FARHAN HERDIAN PRADANA}. Perbandingan Implementasi Algoritma-algoritma Pagerank pada Satu Mesin Komputer. Skripsi. Fakultas Matematika dan Ilmu Pengetahuan Alam, Universitas Negeri Jakarta. 2023. Di bawah bimbingan Muhammad Eka Suryana, M. Kom, dan Med Irzal, M. Kom.

Algoritma Pagerank merupakan algoritma mengurutkan halaman web pada \textit{search engine} Google. Masalah pada Algoritma Pagerank adalah memerlukan memori utama yang cukup besar, dan tidak mungkin dilakukan pada satu mesin komputer dengan memori utama yang terbatas. Akan dicari algoritma alternatif dari Algoritma Pagerank Original buatan Google dengan membandingkannya pada algoritma-algoritma Pagerank dari penelitian lainnya dengan membandingkan kecepatan, penggunaan memori utama, dan kemiripan hasil. Penelitian dilakukan dengan melakukan pengkodean terhadap algoritma Pagerank Original, algoritma Distributed Pagerank Computation (DPC), algoritma Modified DPC (MDPC), dan algoritma Random Walker. Semua kode program dijalankan pada dataset dan dibandingkan kecepatan, penggunaan memori utama, dan kemiripan hasil akhirnya. Khusus hasil akhir, hasil dari algoritma Random Walker dijadikan acuan karena dasar dari Algoritma Pagerank adalah Random Walker. Hasilnya Algoritma Pagerank Original unggul dari sisi kecepatan dan hasil yang mirip dengan hasil Algoritma Random Walker. Sedangkan algoritma DPC dan MDPC unggul di penggunaan memori utama yang lebih hemat, sehingga cocok untuk satu mesin komputer yang memiliki memori utama yang terbatas, tetapi dengan catatan mengorbankan kecepatan yang lebih lambat dan hasil yang tidak mirip terhadap Random Walker.

\noindent
\textbf{Kata Kunci:} \textit{Search engine}, Google, Pagerank, Distributed Pagerank Computation.