%!TEX root = ./template-skripsi.tex
%-------------------------------------------------------------------------------
%                            	BAB IV
%               		KESIMPULAN DAN SARAN
%-------------------------------------------------------------------------------

\chapter{KESIMPULAN DAN SARAN}

\section{Kesimpulan}
Berdasarkan hasil implementasi dan pengujian program Algoritma Pagerank, DPC, MDPC, dan Random Walker, maka diperoleh kesimpulan sebagai berikut:

\begin{enumerate}
	\item Algoritma Pagerank merupakan algoritma untuk menghitung \textit{ranking} halaman web yang berbasis pada Random Walk di graf halaman web \citep{ilprints422}. Algoritma Pagerank memiliki masalah pada penggunaan memori yang besar. Algoritma DPC yang memakai metode \textit{divide and conquer} \citep{zhuetal2005distributedPagerank} dipakai untuk menjawab permasalahan pada algoritma Pagerank. Algoritma MDPC merupakan modifikasi dari algoritma DPC yang dirumuskan pada penelitian ini karena terdapat langkah-langkah yang bisa disederhanakan pada algoritma DPC. Program simulasi Random Walker dibuat untuk menjadi pembanding dari hasil algoritma Pagerank, DPC, dan MDPC.
	
	\item Dari hasil pengujian algoritma pe-\textit{ranking}-an halaman web tercepat dalam waktu eksekusi dipegang oleh algoritma Pagerank, sedangkan dari segi penggunaan memori puncak, MDPC dan DPC jauh lebih kecil dibandingkan algoritma Pagerank. Walaupun demikian, setelah dilakukan uji hasil \textit{ranking} dengan menghitung nilai $KDist$ antara masing-masing algoritma, hasil dari algoritma Pagerank sangat mirip dengan hasil dari algoritma Random Walker dibandingan dengan algoritma DPC, dan MDPC terhadap Random Walker. Sehingga dapat disimpulkan algoritma DPC dan MDPC cocok untuk komputer satu mesin dengan memori utama terbatas, tetapi dengan mengorbankan kemiripan hasil dan waktu eksekusi lebih lambat.
\end{enumerate}

\section{Saran}
Adapun saran untuk penelitian selanjutnya adalah:
\begin{enumerate} 
	\item Menguji keempat algoritma lebih lanjut dengan dataset yang lebih besar dan beragam
    \item Mencari algoritma alternatif lain dari Pagerank yang secara performa memori dan waktu lebih baik, tetapi perbedaan hasil \textit{ranking} di bawah 10 persen
\end{enumerate}


% Baris ini digunakan untuk membantu dalam melakukan sitasi
% Karena diapit dengan comment, maka baris ini akan diabaikan
% oleh compiler LaTeX.
\begin{comment}
\bibliography{daftar-pustaka}
\end{comment}